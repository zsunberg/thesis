\chapter{Introduction}

\section{Autonomous Vehicles}

\subsection{A World with Autonomous Transportation}

\begin{itemize}
    \item Better Safety
    \item More Efficiency
    \item Less wasted time and stress
    \item Better access to transportation
\end{itemize}

\subsection{Current Progress}

(Not sure what to say here)

\subsection{Remaining Challenges}

\subsubsection{Technical}
\subsubsection{Legal and Ethical}
\subsubsection{Business}

\section{Decision Making Under Uncertainty}

\subsection{Uncertainty in Decision Making}

\subsubsection{Outcome Uncertainty}
\subsubsection{Model Uncertainty}
\subsubsection{State Uncertainty}

\todo{POMDP methods also require a reward function.
In some cases, constructing the reward function that balances different objectives (e.g., safety and efficiency) is straightforward. However, in other cases, human preferences can be difficult to quantify and encode into a reward function.
For these cases, inverse reinforcement learning can be used to determine a suitable reward function based on data of how human drivers act \cite{levine2012continuous, sadigh2016leverage}.}


\subsection{Markov Decision Processes}

The Markov decision process (MDP) is a mathematical formalism that can represent a wide range of sequential decision making problems. 
In an MDP, an agent takes \emph{actions} that affect the \emph{state} of the system and collects \emph{rewards} based on the state and actions.
The \emph{Markov property} is a key attribute of MDPs which states that the next state depends (possibly stochastically) on the current state and action, and not on any previous states or actions.

Formally, an MDP is defined by the 5-tuple $(\mathcal{S}, \mathcal{A}, \mathcal{T}, \mathcal{R}, \gamma)$.
The state space, $\mathcal{S}$, is the set of all possible states.
The action space, $\mathcal{A}$, is the set of all actions available to the agent.
The transition model, $\mathcal{T}$, represents likelihood of transitions, where $\mathcal{T}(s' \mid s, a)$ denotes the probability that the system will transition to state $s'$ given that action $a$ is taken in state $s$.
The reward function, $\mathcal{R}: \mathcal{S} \times \mathcal{A} \times \mathcal{S} \to \reals$ represents the rewards received while interacting in the environment, where
$\mathcal{R}(s, a, s')$ denotes the reward for transitioning from $s$ to $s'$ when action $a$ is taken.
Finally, $\gamma$ governs how reward is discounted in the future.

The objective in an MDP is to find a policy, $\pi: \mathcal{S} \to \mathcal{A}$, that maps each encountered state to an action and, when $a_t = \pi(s_t)$, maximizes the cumulative expected reward,
\begin{equation} \label{eq:cumulative}
    E\left[\sum_{t=0}^\infty \gamma^t \mathcal{R}(s_t, a_t, s_{t+1}) \right]
\end{equation}
where the subscript $t$ is the time index. The state action value function, $Q(s,a)$, is defined as the expectation of the future cumulative reward given that the agent starts in state $s$, immediately takes action $a$, and then follows the optimal policy.

\subsection{Partially Observable Markov Decision Processes}

A partially observable Markov decision process (POMDP) is similar to and MDP except that the agent cannot directly observe the state.
Instead, the agent only has access to observations that are generated probabilistically based on the actions and latent true states.
A POMDP is defined by the 7-tuple $(\sspace, \aspace, \mathcal{T}, \mathcal{R}, \ospace, \odist, \gamma)$, where $\sspace$, $\aspace$, $\mathcal{T}$, $\mathcal{R}$, and $\gamma$ have the same meaning as in an MDP.
Additionally, $\ospace$, is the observation space, and $\odist$ is the observation model.
$\odist(o \mid s, a, s')$ is the probability or probability density of receiving observation $o$ in state $s'$ given that the previous state and action were $s$ and $a$.

Information about the state may be inferred from the entire history of previous actions and observations and the initial information, $b_0$.
Thus, in a POMDP, the agent's policy is a function mapping each possible history, $h_t = (b_0, a_0, o_1, a_1, o_2, \dots, a_{t-1}, o_t)$ to an action.
In some cases, each state's probability can be calculated based on the history.
This distribution is known as a \emph{belief}, with $b_t(s)$ denoting the probability of state $s$.

The belief is a sufficient statistic for optimal decision making.
That is, there exists a policy, $\pi^*$ such that, when $a_t = \pi^*(b_t)$, the expected cumulative reward or ``value function'' is maximized for the POMDP \cite{kaelbling1998planning,kochenderfer2015decision}.
Given the POMDP model, each subsequent belief can be calculated using Bayes' rule according to
\begin{equation} \label{eqn:update}
    b'(s') = \frac{\int_{s\in\mathcal{S}} \odist(o \mid s, a, s') \mathcal{T}(s' \mid s, a) b(s) ds}
    {\int_{s'\in\mathcal{S}} \int_{s\in\mathcal{S}} \odist(o \mid s, a, s') \mathcal{T}(s' \mid s, a) b(s) ds ds'} \text{.}
\end{equation}
When the state space is discrete, the integrals may be replaced with sums.

\todo{Talk about PSPACE completeness}

\subsubsection{Generative Models}

or many problems, it can be difficult to explicitly determine or represent the probability distributions $\mathcal{T}$ or $\odist$.
Some solution approaches, however, only require samples from the state transitions and observations.
A generative model, $G$, stochastically generates a new state, reward, and observation in the partially observable case, given the current state and action, that is $s', r = G(s,a)$ for an MDP, or $s', o, r = G(s, a)$ for a POMDP.
A generative model implicitly defines $\mathcal{T}$ and $\odist$, even when they cannot be explicitly represented.

\subsubsection{Belief MDP}

Every POMDP is equivalent to an MDP where the state space of the MDP is the space of possible beliefs.
The reward function of this "belief MDP" is the expectation of the state-action reward function with respect to the belief.
The Bayesian update of the belief serves as a generative model for the belief space MDP.

\subsection{Value Iteration}

\subsection{Monte Carlo Tree Search} \label{sec:mcts}

MCTS is an effective and widely studied algorithms for online decision-making \cite{browne2012survey}.
It works by incrementally creating a policy tree consisting of alternating layers of state nodes and action nodes and estimating the state-action value function, $Q(s,a)$, at each of the action nodes.
Only a generative model, $G$, is required by the algorithm.
The tree is constructed by running $n$ Monte Carlo simulations with four phases, although there are many variations of this algorithm.
\begin{enumerate}
    \item \emph{Search}. In the initial phase of the simulation, the policy defined by the tree is used.
        At each state node, a selection criterion based on $Q$ is used to choose a favorable action, and the tree is traversed through the node and to the next state node determined by $G$.
    \item \emph{Expansion}. Eventually, the simulation reaches an action node that does not have any children.
        At this point, a new state is sampled with $G$ and a corresponding node created along with children corresponding to each action. 
    \item \emph{Rollout}. After the expansion step, the simulation is continued with a rollout policy, often consisting of randomly selected actions, until the future accumulated reward will be negligible because of the compounding discount factor.
    \item \emph{Update}. Once the simulation has terminated, the estimates of $Q(s,a)$ at each of the visited action nodes are updated with the discounted reward received during the simulation after visiting the node.
\end{enumerate}
This approach builds a tree asymmetrically favoring regions of the state and action spaces that will be visited when the optimal policy is executed.

\subsubsection{Upper Confidence Trees} \label{sec:uct}

The selection criterion used to choose actions in the search phase is very important.
It must balance favoring actions with large $Q$ values that are expected to yield good results with exploration of new actions.
The most widely used approach for this is known as the upper confidence bound for trees (UCT) algorithm.
At each state node, it chooses the action that maximizes the upper confidence bound
\begin{equation} \label{eqn:ucb}
    UCB(s,a) = Q(s,a) + c \sqrt{\frac{\log N(s)}{N(s,a)}}
\end{equation}
where $N(s,a)$ is the number of times the action node has been visited, $N(s) = \sum_{a \in \mathcal{A}} N(s,a)$, and $c$ is a problem-specific parameter that governs the amount of exploration in the tree.
The second term causes the algorithm to favor actions that have been taken less often.

\subsubsection{Double Progressive Widening} \label{sec:dpw}

In cases where the action and state spaces are large or continuous, the MCTS algorithm will produce trees that are very shallow.
In fact, if the action space is continuous, the UCT algorithm will never try the same action twice (observe that, if $N(s,a) = 0$ then $UCB(s,a)$ in (\ref{eqn:ucb}) is infinite, so untried actions are always favored).
Moreover, if the state space is continuous and the transition probability density is finite, the probability of sampling the same state twice from $G$ is zero.
Because of this, simulations will never pass through the same state node twice and a tree below the first layer of state nodes will never be constructed.

In progressive widening, the number of children of a node is artificially limited to $k N^\alpha$ where $N$ is the number of times the node has been visited and $k$ and $\alpha$ are parameters chosen for the problem \cite{couetoux2011double}.
Originally, progressive widening was applied to the action space, and was found to be especially effective when a set of preferred actions was tried first \cite{browne2012survey}.
The term \emph{double} progressive widening refers to progressive widening in both the state and action space.
When the number of state nodes is greater than the limit, instead of simulating a new state transition, one of the previously generated states is chosen with probability proportional to the number of times it has been previously generated.


\subsection{Particle Filtering} \label{sec:particle}

Aside from a few special cases, for example when the system dynamics are linear and the transition and observation distributions are Gaussian, the integrals in the Bayesian belief update (\ref{eqn:update}) are impossible or difficult to solve analytically.
Thus, numerical approaches must be used.
A popular technique for this is particle filtering, usually incorporating domain-specific heuristic modifications to prevent problems such as particle depletion \cite{thrun2005probabilistic}.

Sequential importance resampling, one of the most common and effective variations, requires a state generative model, $G_s$ that can sample from the transition distribution and an explicit representation of the observation distribution, $\odist(\cdot \mid s, a, s')$, which is often easier to represent than the transition distribution.
The belief is approximated with a set of $m$ particles, $\{s_i\}_{i=1}^m$ and associated weights, $\{w_i\}_{i=1}^m$.
The probability of each state is approximated by the sum of the weights corresponding to particles with that state value,
\begin{equation}
    b(s) \approx \sum_{i=1}^m w_i \delta_s (s_i)
\end{equation}
where $\delta_s(\cdot)$ is the Dirac delta function centered at $s$.
A belief update is approximated by sampling $m$ states $\{\tilde{s}_i\}_{i=1}^m$ from the collection of particles with probability proportional to the associated weight, generating a particle, $s'_i = G(\tilde{s}_i, a)$ for each of these states, and finally setting the new weight proportional to the probability of generating the measured observation with this state, $w'_i \propto \odist(o \mid \tilde{s}_i, a, s'_i)$.

\subsection{Approximate Solutions to POMDPs} \label{sec:solutions}

Considerable progress has been made in solving large POMDPs.
Initially, exact offline solutions to problems with only a few discrete states, actions, and observations were sought by using value iteration and taking advantage of the convexity of the value function \cite{kaelbling1998planning}, although solutions to larger problems were also explored using Monte Carlo simulation and interpolation between belief states \cite{thrun1999monte}.
Many effective offline planners for discrete problems use point based value iteration, where a selection of points in the belief space are used for value function approximation,  \cite{kurniawati2008sarsop}.
Offline solutions for problems with continuous state and observation spaces have also been proposed \cite{bai2014integrated,brechtel2013solving}.

There are also various solution approaches that are applicable to specific classes of POMDPs, including continuous problems.
For example, \citet{platt2010belief} simplify planning in large domains by assuming that the most likely observation will always be received, which can provide an acceptable approximation in some problems with unimodal observation distributions.
\citet{morere2016bayesian} solve a monitoring problem with continuous spaces with a Gaussian process belief update.
\citet{hoey2005solving} propose a method for partitioning large observation spaces without information loss, but demonstrate the method only on small state and action spaces that have a modest number of conditional plans.
Other methods involve motion-planning techniques \cite{melchior2007particle,prentice2009belief,bry2011rapidly}.
In particular, \citet{agha2011firm} present a method to take advantage of the existence of a stabilizing controller in belief space planning.
\citet{van2012motion} perform local optimization with respect to uncertainty on a pre-computed path, and \citet{indelman2015planning} devise a hierarchical approach that handles uncertainty in both the robot's state and the surrounding environment.

General purpose online algorithms for POMDPs have also been proposed.
These algorithms are mostly derivatives of the MCTS algorithm for MDPs (\cref{sec:mcts}).
A conceptually straightforward way to solve a POMDP using MCTS is to apply it to the corresponding belief MDP.
Indeed, many tree search techniques have been applied to POMDP problems in this way \cite{ross2008online}.
However, when the Bayesian belief update is used, this approach is computationally expensive.
POMCP can tackle problems many times larger than its predecessors because it uses state trajectory simulations, rather than full belief trajectories, to build the tree.
Determinized sparse partially observable tree (DESPOT) is a similar approach that attempts to achieve better performance by analyzing only a small number of random outcomes in the tree \cite{somani2013despot}.
Adaptive belief tree (ABT) was designed specifically to accommodate changes in the environment without having to replan from scratch \cite{kurniawati2016online}.

\todo{Add POMCP section(?), or at least specifics}

\subsection{QMDP} \label{sec:qmdp}

\todo{write QMDP section}

\section{Contributions}
