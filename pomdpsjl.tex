\chapter{POMDPs.jl: A Framework for Sequential Decision Making under Uncertainty} \label{chap:pomdpsjl}

This chapter describes the POMDPs.jl software package created by the Stanford Intelligent Systems Lab (SISL) to make state-of-the-art POMDP solution methods easily accessible to students, researchers, and engineers.
All of the research in \cref{chap:multilane,chap:pomcpow} was conducted using this framework.

\todo{Use cases?}

\section{Challenges for POMDP-solving software}

A successful POMDP software framework must have, at a minimum, the following attributes: speed, flexibility, and ease of use.

\subsection{Speed}

Since POMDPs are difficult to solve\todo{Make stronger statement about computational complexity}, any computational slowdown such as unnecessary memory allocation or runtime type inference significantly reduces the maximum problem size that the framework can handle.
For this reason, POMDP algorithms must be compiled to efficient processor instructions with low overhead.

\subsection{Flexibility}

The set of problems that can be represented as a POMDP is extremely large and there are many possible characteristics that such problems might have.
A good POMDP software framework should try to accommodate as much of this set as possible.
A few of the most important model characteristics to support are outlined below.

\subsubsection{Partial and full observability}

When studying a POMDP problem, it is almost always important to analyze the underlying fully-observable problem.
Thus, a good POMDP framework should have first-class support for MDPs in addition to POMDPs.

\subsubsection{Continuous and discrete problems}

Some POMDPs have a finite number of states, actions, and observations, i.e. $|\sspace| < \infty$, $|\aspace| < \infty$, and $|\ospace| < \infty$.
However, many real world problems, notably robotics problems, are naturally formulated in spaces with uncountably infinite cardinality, e.g. $\sspace = \reals$, $\aspace = \reals$, and $\ospace = \reals$, or multi-dimensional vector spaces, e.g. $\sspace = \reals^6$
This means that the framework cannot be constrained to use integers for state representation, but should be capable of using floating point vectors.

\subsubsection{Explicit vs generative model representation}

The explicit

\subsubsection{Online and offline solvers}

While some POMDP solution techniques seek exact offline solutions to small problems, many larger problems can only be practically solved online.
Thus a good POMDP framework must have first class support for both solving offline and efficiently executing a policy online or executing a planner that does significant computation online.

\subsubsection{Policy representation}

The policies that 

\subsubsection{Exact and approximate solutions}

\subsection{Ease of Use}


\section{Previous frameworks}

\section{Architecture}

\subsection{Concepts}

\subsection{Interfaces}

\section{Examples}
