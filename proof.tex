\chapter{Proof of Theorem 1} \label{sec:proof}

A version of Monte Carlo tree search with double progressive widening has been proven to converge to the optimal value function on fully observable MDPs by \citet{auger2013continuous}.
We use this proof to show that POMCP-DPW converges to the QMDP solution.

First we establish some preliminary definitions taken directly from \citet{auger2013continuous}.

\begin{definition}[Regularity Hypothesis]
    The \emph{Regularity hypothesis} is the assumption that for any $\Delta > 0$, there is a non zero probability to sample an action that is optimal with precision $\Delta$. More precisely, there is a $\theta > 0$ and a $p > 1$ (which remain the same during the whole simulation) such that for all $\Delta > 0$, 
\begin{align}
    Q(ha) \geq Q^*(h)-\Delta \text{ with probability at least } \min(1, \theta \Delta^p)\text{.}
\end{align}
\end{definition}

\begin{definition}[Exponentially sure in $n$]
    We say that some property depending on an integer $n$ is exponentially sure in $n$ if there exists positive constants $C$, $h$, and $\eta$ such that the probability that the property holds is at least $$1-C \exp(-hn^\eta)\text{.}$$
\end{definition}

In order for the proof from \citet{auger2013continuous} to apply, the following four minor modifications to the POMCP-DPW algorithm must be made: 

\begin{enumerate}
    \item Instead of the usual logarithmic exploration, use \emph{polynomial exploration}, that is, select actions based on the criterion
    \begin{equation}
        Q(ha) + \sqrt{\frac{N(h)^{e_d}}{N(ha)}}\text{,}
    \end{equation}
    as opposed to the traditional criterion
    \begin{equation}
        Q(ha) + c \sqrt{\frac{\log N(h)}{N(ha)}}\text{,}
    \end{equation}
    and create a new node for progressive widening when $\lfloor N^\alpha \rfloor > \lfloor (N-1)^\alpha \rfloor$ rather than when the number of children exceeds $k N^\alpha$.

    \item Instead of performing rollout simulations, keep creating new single-child nodes until the maximum depth is reached.

    \item In line~\ref{lin:selecto}, instead of selecting an observation randomly, select the observation that has been visited least proportionally to how many times it has been generated.

    \item Use the depth-dependent coefficient values in Table 1 from \citet{auger2013continuous} instead of choosing static values.
\end{enumerate}

This version of the algorithm will be referred to as ``modified POMCP-DPW''. The algorithm with these changes is listed in \Cref{alg:mpomcpdpw}.

\begin{algorithm}[htb]
    \caption{Modified POMCP-DPW} \label{alg:mpomcpdpw}
    \begin{algorithmic}[1]
        \Procedure{Plan}{$b$}
            \For{$i \in 1:n$}
                \State $s \gets \text{sample from }b$ \label{lin:msample}
                \State $\Call{Simulate}{s, b, d_\text{max}}$
            \EndFor
            \State $\textbf{return } \underset{a}{\argmax}\, Q(ba)$
        \EndProcedure

        \Procedure {ActionProgWiden}{$h$}
            \If{$\lfloor N(h)^{\alpha_{a,d}} \rfloor > \lfloor (N(h)-1)^{\alpha_{a,d}} \rfloor$}
                \State $a \gets \Call{NextAction}{h}$
                \State $C(h) \gets C(h) \cup \{a\}$
            \EndIf
            \State $\textbf{return } \underset{a \in C(h)}{\argmax}\, Q(ha) + \sqrt{\frac{N(h)^{e_d}}{N(ha)}}$
        \EndProcedure

        \Procedure {Simulate}{$s$, $h$, $d$}        
            \If{$d = 0$}
                \State \textbf{return} $0$
            \EndIf
            \State $a \gets \Call{ActionProgWiden}{h}$
            \If{$\lfloor N(ha)^{\alpha_{o,d}} \rfloor > \lfloor (N(ha)-1)^{\alpha_{o,d}} \rfloor$}
                \State $s',o,r \gets G(s,a)$
                \State $C(ha) \gets C(ha) \cup \{o\}$
                \State $M(hao) \gets M(hao) + 1$ \label{lin:gencount}
                \State $\text{append } s' \text{ to } B(hao)$ \label{lin:minsertion}
            \Else
                \State $o \gets \underset{o \in C(ha)}{\argmin}\, N(hao)/M(hao)$
                \State $s' \gets \text{select } s' \in B(hao) \text{ w.p. } \frac{1}{|B(hao)|}$
                \State $r \gets \reward(s,a,s')$
            \EndIf
            \State $total \gets r + \gamma \Call{Simulate}{s', hao, d-1}$
            \State $N(h) \gets N(h)+1$
            \State $N(ha) \gets N(ha)+1$
            \State $Q(ha) \gets Q(ha) + \frac{total - Q(ha)}{N(ha)}$
            \State \textbf{return} $total$
        \EndProcedure
    \end{algorithmic}        
\end{algorithm}

We now define the ``QMDP value'' that POMCP-DPW converges to (this is repeated from the main text of the paper) and prove a preliminary lemma.

\qmdpvalue*
% \begin{definition}[QMDP value]
%      Let $Q_\text{MDP}(s,a)$ be the optimal state-action value function assuming full observability starting by taking action $a$ in state $s$.
%      The \emph{QMDP value} at belief $b$, $Q_\text{MDP}(b,a)$, is the expected value of $Q_\text{MDP}(s,a)$ when $s$ is distributed according to $b$.   
% \end{definition}

\begin{lemma} \label{lem:onestate}
    If POMCP-DPW or modified POMCP-DPW is applied to a POMDP with a continuous observation space and observation probability density functions that are finite everywhere, then each history node in the tree will have only one corresponding state, that is $|B(h)| = 1, M(h)=1\, \forall h$.
\end{lemma}

\begin{proof}
    Since the observation probability density function is finite, each call to the generative model will produce a unique observation with probability 1.
    Because of this, lines~\ref{lin:gencount}~and~\ref{lin:minsertion} of \cref{alg:mpomcpdpw} will only be executed once for each observation.
\end{proof}

We are now ready to restate the theorem from the text.

% \begin{theorem}[Modified POMCP-DPW convergence to QMDP] \label{thm:pqmdp}
% If a bounded-horizon POMDP meets the following conditions: 1) the state and observation spaces are continuous with a finite observation probability density function, and 2) the regularity hypothesis is met, then modified POMCP-DPW will produce a value function estimate, $\hat{Q}$, that converges to the QMDP value for the problem.
% Specifically, there exists a constant $C>0$, such that after $n$ iterations,
% \begin{equation*}
%     \left| \hat{Q}(b,a) - Q_\text{MDP}(b,a) \right| \leq \frac{C}{n^{1/(10d_{\max}-7)}}
% \end{equation*}
% exponentially surely in $n$, for every action $a$.
% \end{theorem}

\qmdp*

The bound on the value estimate error, $\frac{C}{n^{1/(10d_{\max}-7)}}$,
% \begin{equation}
%     \frac{C}{n^{1/(10d_{\max}-7)}}
% \end{equation}
is based on the specific coefficients chosen by \citet{auger2013continuous} and listed in Table 1 of their paper. Alternative bounds may be possible with different coefficient choices. The proof of the theorem is given below.

\begin{proof}
    We prove that modified POMCP-DPW functions exactly as the Polynomial UCT (PUCT) algorithm defined by \citet{auger2013continuous} applied to an augmented fully observable MDP, and hence converges to the QMDP value.
    We will show this by proposing incremental changes to \cref{alg:mpomcpdpw} that do not change its function that will result in an algorithm identical to PUCT.

    Before listing the changes, we define the ``augmented fully observable MDP" as follows: For a POMDP $\mathcal{P} = (\mathcal{S}, \mathcal{A}, \mathcal{T}, \mathcal{R}, \mathcal{O}, \mathcal{Z}, \gamma)$, and belief $b$, the \emph{augmented fully observable MDP}, $\mathcal{M}$, is the MDP defined by $(\mathcal{S}_A, \mathcal{A}, \mathcal{T}_A, \mathcal{R}, \gamma)$, where 
    \begin{equation}
        \mathcal{S}_A = \mathcal{S} \cup \{b\}
    \end{equation}
    and, for all $x, x' \in \mathcal{S}_A$,
    \begin{equation}
        \mathcal{T}_A (x'|x, a) = \begin{cases}
                \mathcal{T} (x' | x, a) & \text{if } x \in \mathcal{S} \\
                \int_S b(s) \mathcal{T} (x' | s, a) ds & \text{if } x = b
        \end{cases}
    \end{equation}
    This is simply the fully observable MDP augmented with a special state representing the current belief.
    It is clear that the value function for this problem, $Q_\mathcal{M}(b, a)$, is the same as the QMDP value for the POMDP, $Q_\text{MDP}(b,a)$.
    Thus, by showing that modified POMCP-DPW behaves exactly as PUCT applied to $\mathcal{M}$, we show that it estimates the QMDP values.

    % For the above definition, we assume the state space is integrable. Is that ok without stating?

    Consider the following modifications to \cref{alg:mpomcpdpw} that do not change its behavior when the observation space is continuous:

    \begin{enumerate}
        \item Eliminate the state count $M$. \emph{Justification}: By \cref{lem:onestate}, its value will be 1 for every node.
        \item Remove $B$ and replace with a mapping $H$ from each node to a state of $\mathcal{M}$; define $H(b) = b$. \emph{Justification}: By \cref{lem:onestate}, $B$ always contains only a single state, so $H$ contains the same information.
        \item Generate states and rewards with $G_\mathcal{M}$, the generative model of $\mathcal{M}$, instead of $G$. \emph{Justification}: Since the state transition model for the fully observable MDP is the same as the POMDP, these are equivalent for all $s \in \mathcal{S}$.
        \item Remove the $s$ argument of \textproc{Simulate}. \emph{Justification}: The sampling in line~\ref{lin:msample} is done implicitly in $G_\mathcal{M}$ if $h=b$, and $s$ is redundant in other cases because $h$ can be mapped to $s$ through $H$.
    \end{enumerate}

    The result of these changes is shown in \cref{alg:c}. It is straightforward to verify that this algorithm is equivalent to PUCT applied to $\mathcal{M}$.
    Each observation-terminated history, $h$, corresponds to a PUCT ``decision node'', $z$, and each action-terminated history, $ha$, corresponds to a PUCT ``chance node'', $w$.
    In other words, the observations have no meaning in the tree other than making up the histories, which are effectively just keys to or aliases for the state nodes.
    
    Since PUCT is guaranteed by Theorem 1 of \citet{auger2013continuous} to converge to the optimal value function of $\mathcal{M}$ exponentially surely, POMCP-DPW is guaranteed to converge to the QMDP value exponentially surely, and the theorem is proven.

\begin{algorithm}[htb]
    \caption{Modified POMCP-DPW on a continuous observation space} \label{alg:c}
    \begin{algorithmic}[1]
        \Procedure{Plan}{$b$}
            \For{$i \in 1:n$}
                \State $\Call{Simulate}{(b), d_\text{max}}$
            \EndFor
            \State $\textbf{return } \underset{a}{\argmax}\, Q(ha)$
        \EndProcedure

        \Procedure {ActionProgWiden}{$h$}
            \If{$\lfloor N(h)^{\alpha_{a,d}} \rfloor > \lfloor (N(h)-1)^{\alpha_{a,d}} \rfloor$}
                \State $a \gets \Call{NextAction}{h}$
                \State $C(h) \gets C(h) \cup \{a\}$
            \EndIf
            \State $\textbf{return } \underset{a \in C(h)}{\argmax}\, Q(ha) + \sqrt{\frac{N(h)^{e_d}}{N(ha)}}$
        \EndProcedure

        \Procedure {Simulate}{$h$, $d$}        
            \If{$d = 0$}
                \State \textbf{return} $0$
            \EndIf
            \State $a \gets \Call{ActionProgWiden}{h, d}$
            \If{$\lfloor N(ha)^{\alpha_{o,d}} \rfloor > \lfloor (N(ha)-1)^{\alpha_{o,d}} \rfloor$}
                \State $\cdot, o, \cdot \gets G(H(h), a)$
                \State $H(hao),r \gets G_\mathcal{M}(H(h),a)$
                \State $C(ha) \gets C(ha) \cup \{o\}$
            \Else
                \State $o \gets \underset{o \in C(ha)}{\argmin}\, N(hao)$
                \State $r \gets \reward(H(h), a, H(hao))$
            \EndIf
            \State $total \gets r + \gamma \Call{Simulate}{hao, d-1}$
            \State $N(h) \gets N(h)+1$
            \State $N(ha) \gets N(ha)+1$
            \State $Q(ha) \gets Q(ha) + \frac{total - Q(ha)}{N(ha)}$
            \State \textbf{return} $total$
        \EndProcedure
    \end{algorithmic}
\end{algorithm}

\end{proof}

\begin{remark}
    One may object that multiple histories may map to the same state through $H$, and thus the history nodes in a modified POMCP-DPW tree are not equivalent to state nodes in the PUCT tree. In fact, the PUCT algorithm does not check to see if a state has previously been generated by the model, so it may also contain multiple decision nodes $z$ that correspond to the same state. Though this is not explicitly stated by the authors, it is clear from the algorithm description, and the proof still holds.
\end{remark}
